\documentclass[12pt,a4paper,oneside]{report}

\usepackage{titlesec}
\usepackage{geometry}
\usepackage{graphicx}
\usepackage{booktabs, chemformula}
\usepackage{titlesec, blindtext, color}
\usepackage{listings}
\usepackage{natbib}
\usepackage{xcolor} % for setting colors

% set the default code style
\lstset{
basicstyle=\fontsize{9}{11}\ttfamily,
    frame=tb, % draw a frame at the top and bottom of the code block
    tabsize=4, % tab space width
    showstringspaces=false, % don't mark spaces in strings
    numbers=left, % display line numbers on the left
    commentstyle=\color{green}, % comment color
    keywordstyle=\color{blue}, % keyword color
    stringstyle=\color{red} % string color
}

\newcommand\tab[1][1cm]{\hspace*{#1}}

 \geometry{
 a4paper,
 top=35mm,
bottom=30mm,
bindingoffset=0.0in
 }
\pagenumbering{arabic} 

\pagestyle{headings}

\newcommand{\dspaceon}{\renewcommand{\baselinestretch}{1.3}\large\normalsize}
\newcommand{\dspaceoff}{\renewcommand{\baselinestretch}{1}\large\normalsize}

\graphicspath{ {images/} }

\definecolor{gray75}{gray}{0.75}
\newcommand{\hsp}{\hspace{20pt}}
\titleformat{\chapter}[hang]{\Large\bfseries}{\thechapter\hsp\textcolor{gray75}{$|$}\hsp}{0pt}{\Large\bfseries}
\titleformat{\section}{\normalfont\large\bfseries}{\thesection}{1em}{}

\begin{document}
{\let\clearpage\relax \chapter{Presentation of the research topic}}
The domain of the proposed thesis is Automated Software Engineering. The thesis will develop methods for the analysis of legacy software systems, focusing on using historical information describing the evolution of the systems extracted from the versioning systems. 
The methods for analysis will integrate techniques based on computational algorithms as well as data-mining. As proof-of-concept, tool prototypes will implement the proposed methods and validate them by extensive experimentation on several cases of real-life systems.

{\let\clearpage\relax \chapter{Current status of research within the proposed topic}}
The current trend recommends that general dependency management methods and tools should also include logical dependencies besides the structural dependencies \cite{Oliva:2011:ISL:2067853.2068086}, \cite{DBLP:journals/jss/AjienkaC17}. 

A dependency is created by two elements that are in a relationship and indicates that an element of the relationship, in some manner, depends on the other element of the relationship \cite{Booch:2004:OAD:975416}, \cite{Cataldo2009SoftwareDW}. In the case of object oriented software systems, dependency models are usually class dependency models where elements are entities such as classes and interfaces \cite{Sangal:2005:UDM:1094811.1094824}. There are several types of relationships between these source code entities, for example a method of a class can call a method of another class, a class extends another class, all those create  \textit{structural dependencies} between classes (a.k.a syntactic dependencies or structural coupling). These dependencies can be found by the analysis of the source code.

Software engineering practice has shown that sometimes modules which do not present structural dependencies still appear to be related. Co-evolution represents the phenomenon when one component changes in response to a change in another component \cite{Yu:2007:UCC:1231330.1231370}. Those changes can be found in the software history maintained by the versioning system. Gall \cite{Gall:1998:DLC:850947.853338} identified as logical coupling between two modules the fact that these modules  \textit{repeatedly} change together during the historical evolution of the software system. 
 \textit{Logical dependencies} (a.k.a logical coupling) can be found by software history analysis and can reveal relationships that are not always present in the source code (structural dependencies).  


%Structural dependencies are the result of the source code analysis of the system \cite{struct_dep} while logical dependencies refer to those dependencies between entities that are not always visible through source code analysis and can be added during the development process. Logical dependencies can be easily extracted from the versioning system (e.g. Subversion , Git) commits. 

The concepts of logical coupling and logical dependencies were first used in different analysis tasks, all related to changes: for software change impact analysis \cite{1553643}, for identifying the potential ripple effects caused by software changes during software maintenance and evolution \cite{DBLP:conf/issre/OlivaG15}, \cite{Oliva:2011:ISL:2067853.2068086}, \cite{Poshyvanyk2009}, \cite{posh2010} or for their link to deffects \cite{wiese}, \cite{Zimmermann:2004:MVH:998675.999460}.

Different applications based on dependency analysis could be improved if, beyond structural dependencies, they also take into account the hidden non-structural dependencies. For example, works  which investigate different methods for architectural reconstruction \cite{SoraConti}, \cite{SoraSem13}, \cite{PagerankENASE},  all of them based on the information provided by structural dependencies, could enrich their dependency models by taking into account also logical dependencies. However, a thorough survey \cite{sar} shows that historical information has been rarely used in architectural reconstruction. 

Another survey \cite{Shtern:2012:CMS:2332427.2332428} mentions one possible explanation why historical information have been rarely used in architectural reconstruction: the size of the extracted information. One problem is the size of the extraction process, which has to analyze many versions from the historical evolution of the system. Another problem is the big number of pairs of classes which record co-changes and how they relate to the number of pairs of classes with structural dependencies.


{\let\clearpage\relax \chapter{Justification of research topic}}
The software architecture is important in order to understand and maintain a system. Often code updates are made without checking or updating the architecture. This kind of updates cause the architecture to drift from the reality of the code over time.\cite{sar}
So reconstructing the architecture and verifying if still matches the reality is important. 

Surveys show that architectural reconstruction is mainly made based on structural dependencies \cite{Shtern:2012:CMS:2332427.2332428} \cite{sar}, the main reason why historical information is rarely used in architectural reconstruction is the size of the extracted information.

Logical dependencies should integrate harmoniously with structural dependencies in an unitary dependency model: valid logical dependencies should not be omitted from the dependency model, but structural dependencies should not be engulfed by questionable logical dependencies generated by casual co-changes.  
Thus, in order to add logical dependencies besides structural dependencies in dependency models, class co-changes must be filtered until they remain only a reduced but relevant set of valid logical dependencies. 

Currently there is no set of rules or best practices that can be applied to the extracted class co-changes and can guarantee their filtering into a set of valid logical dependencies.
This is mainly because not all the updates made in the versioning system are code related. For example a commit that has as participants a big number of files can indicate that a merge with another branch or a folder renaming has been made. In this case, a series of irrelevant co-changing pairs of entities can be introduced. So, in order to exclude this kind of situations the information extracted from the versioning system has to be filtered first and then used.

Other works have tried to filter co-changes \cite{Oliva:2011:ISL:2067853.2068086}, \cite{DBLP:journals/jss/AjienkaC17}. One of the used co-changes filter is the commit size.The commit size is the number of code files changed in that particular commit. 
Ajienka and Capiluppi established a threshold of 10 for the maximum accepted size for a commit \cite{DBLP:journals/jss/AjienkaC17}. This means that all the commits that had more than 10 code files changed where discarded from the research. But setting a harcoded threshold for the commit size is debatable because in order to say that a commit is big or small you have to look first at the size of the system and at the trends from the versioning system. Even thought the best practices encourage small and often commits, the developers culture is the one that influences the most the trending size of commits from one system.

Filtering only after commit size is not enought, this type of filtering can indeed have an impact on the total number of extracted co-changes, but will only shrink the number of co-changes extracted without actually guaranteeing that the remaining ones have more relevancy and are more logical linked.

Although, some unrelated files can be updated by human error in small commits, for example: one file was forgot to be commited in the current commit and will be commited in the next one among some unrelated files. This kind of situation can introduce a set of co-changing pairs that are definetly not logical liked. In order to avoid this kind of situation a filter for the occurrence rate of co-changing pairs must be introduced. Co-changing pairs that occur multiple times are more prone to be logically dependent than the ones that occur only once. Currently there are no concrete examples of how the threshold for this type of filter can be calculated. In order to do that, incrementing the threshold by a certain step will be the start and then studying the impact on the remaining co-changing pairs for different systems. 

Taking into account also structural dependencies from all the revisions of the system was not made in previous works, this step is important in order to filter out the old, out-of-date logical dependencies. Some logical dependencies may have been also structural in previous revisions of the system but not in the current one. If we take into consideration also structural dependencies from previous revisions then the overlapping rate between logical and structural dependencies could probably increase. Another way to investigate this problem could be to study the trend of concurrences of co-changes: if co-changes between a pair of classes used to happen more often in the remote past than in the more recent past, it may be a sign that the problem causing the logical coupling has been removed in the mean time. 

Also, logical dependency can be also a structural dependency and vice-versa, so studying the overlapping between logical and structural dependencies while filtering is important since the intention is to introduce those logical dependencies among with structural dependencies in architectural reconstruction systems. Current studies have shown a relatively small percentage of overlapping between them with and without any kind of filtering \cite{DBLP:journals/jss/AjienkaC17}. This means that a lot of non related entities update together in the versioning system, the goal here is to establish the factors that determine such a small percentage of overlapping.
 


{\let\clearpage\relax \chapter{Research content and stages of research-implementation schedule}}

The research will be made by following the next stages of implementation:

\textbf{Stage 1:} Build tool to extract structural dependencies from code and co-changes from git for a given set of projects.

\textbf{Stage 2:} Find filters for the co-changes extracted, the filters can be the ones already mentioned in previous works or new ones. Establish different thresholds for those filters.

\textbf{Stage 3:} Study the impact of those filters and the coresponding thresholds on the remaining quantity of co-changes for each system.
Study the overlappings between the remaining pairs of co-changing entites and the structural dependencies extracted.

\textbf{Stage 4:} Based on the findings from stage 3, establish a dynamic method to determine the thresholds in order to fit the best each studied sistem.

\textbf{Stage 5:} Export the remaining co-changes whom at this step we can call logical dependencies and use them among structural dependencies in tools for architectural reconstruction to evaluate the improvement.
Also manually evaluating some logical dependencies extracted from the systems in order to verify their true logical connection.

\textbf{Stage 6:} Identify or build other tools that use historical information and evaluate the impact of filtering of co-changes into logical dependences for them. 

\chapter{Necessary resources and available resources in UPT for the implementation of the research training}

The topic will be developed within the Database and Artificial Intelligence lab from UPT. The topic continues the research directions from grant "Automated recovery of architectural information from source code - AReAS".

\bibliographystyle{plain}
\bibliography{logicaldepd}


\end{document}